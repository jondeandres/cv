%%%%%%%%%%%%%%%%%%%%%%%%%%%%%%%%%%%%%%%%%
% Jon de Andres Frias CV
% jon.deandres@gmail.com
% https://github.com/jondeandres
%
% This template has been taken from
% http://www.latextemplates.com/template/classicthesis-styled-cv
% and odified by me to fit my needs
%%%%%%%%%%%%%%%%%%%%%%%%%%%%%%%%%%%%%%%%%%

%----------------------------------------------------------------------------------------
%	PACKAGES AND OTHER DOCUMENT CONFIGURATIONS
%----------------------------------------------------------------------------------------

\documentclass{scrartcl}
\reversemarginpar % Move the margin to the left of the page 

\newcommand{\MarginText}[1]{\marginpar{\raggedleft\itshape\small#1}} % New command defining the margin text style

\usepackage[nochapters]{classicthesis} % Use the classicthesis style for the style of the document
\usepackage[LabelsAligned]{currvita} % Use the currvita style for the layout of the document

\renewcommand{\cvheadingfont}{\LARGE\color{NavyBlue}} 

\usepackage{hyperref} % Required for adding links	and customizing them
\hypersetup{colorlinks, breaklinks, urlcolor=NavyBlue, linkcolor=NavyBlue} % Set link colors

\newlength{\datebox}\settowidth{\datebox}{Spring 2011} % Set the width of the date box in each block

\newcommand{\NewEntry}[3]{\noindent\hangindent=2em\hangafter=0 \parbox{\datebox}{\small \textit{#1}}\hspace{1.5em} #2 #3 % Define a command for each new block - change spacing and font sizes here: #1 is the left margin, #2 is the italic date field and #3 is the position/employer/location field
\vspace{0.5em}} % Add some white space after each new entry

\newcommand{\Description}[1]{\hangindent=2em\hangafter=0\noindent\raggedright\footnotesize{#1}\par\normalsize\vspace{1em}} % Define a command for descriptions of each entry - change spacing and font sizes here

%----------------------------------------------------------------------------------------

\begin{document}

\thispagestyle{empty} % Stop the page count at the bottom of the first page

%----------------------------------------------------------------------------------------
%	NAME AND CONTACT INFORMATION SECTION
%----------------------------------------------------------------------------------------

\begin{cv}{\spacedallcaps{Jon de Andres Frias}}\vspace{1.5em} % Your name

\noindent\spacedlowsmallcaps{Personal Information}\vspace{0.5em} % Personal information heading

\NewEntry{}{\textit{Born in Vitoria-Gasteiz,}}{22 January 1984} % Birthplace and date

\NewEntry{email}{\href{mailto:jon.deandres@gmail.com}{jon.deandres@gmail.com}} % Email address

\NewEntry{github}{\href{http://github.com/jondeandres}{http://github.com/jondeandres}} % Personal website

\NewEntry{website}{\href{http://jondeandres.github.io}{http://jondeandres.github.io}} % Personal website

\NewEntry{phone}{(M) +34 635730544} % Phone number(s)

\vspace{1em} % Extra white space between the personal information section and goal

\noindent\spacedlowsmallcaps{Goal}\vspace{1em} % Goal heading, could be used for a quotation or short profile instead

\Description{I like to code and to be near people, coders or not, to
  learn from. One of the most important thing in a job for me is the
  team and i'd like to find that team to work with. Code and design
  software architectures is what i like and i hope to find an interesting project to do this.}\vspace{2em} % Goal text

%----------------------------------------------------------------------------------------
%	WORK EXPERIENCE
%----------------------------------------------------------------------------------------

\noindent\spacedlowsmallcaps{Work Experience}\vspace{1em}

\NewEntry{2012--Present}{Software Engineer, \textsc{Wuaki.tv}
  --- Barcelona}

\Description{\MarginText{Wuaki.tv}Wuaki.tv offers a video on demand service with movies and tv shows. I work in the backend department as ruby developer and i'm the tech lead of one of the development
  teams.}
\Description{\MarginText{}In the backend department we work with Ruby on
  Rails and Sinatra as web frameworks. Most of the applications are
  written using Rails but we developed a independient user service
  using Sinatra.}
\Description{\MarginText{}I was one of the members of the team that developed the project to
  launch Wuaki.tv in UK, and the mentioned user service was one of the
  most important pieces in this project. A very critical task in the
  project was the integration with play.com (Play.com). Play.com users
  are now able to sign-in in Wuaki.tv UK with their Play.com
  credentials and all their data is synced}
\Description{\MarginText{}The team i'm leading is developing a project
  with Grape (similar to Sinatra), RabbitMQ and Riak
  to offer an asyncronous and common payment gateway for the different
  applications, platforms and countries we support.}
\Description{\MarginText{}Other technologies we use are Redis, MySQL,
  Capistrano, Janky, Jenkins, New Relic and Errbit(Airbrake).
  \\ \href{http://es.wuaki.tv}{http://es.wuaki.tv}}
%------------------------------------------------

\NewEntry{2011--2012}{Software Engineer, \textsc{Gnuine}  --- Barcelona}

\Description{\MarginText{Gnuine}Gnuine it's a company in Barcelona that develops
    project for third party companies. Gnuine
    develops its own web framework, Ubiquo, on top of Ruby on Rails and most of
    the developments are done with this framework.}
\Description{
    In these projects usually we developed the backend with ruby and
    most part of the javascript development in the frontend. At the
    end of my time in Gnuine i was lead project developer and projects manager.
    The most important projects i worked for are:
    \begin{itemize}
    \item FC Barcelona. I developed some parts of the Ajax widgets for
      this projects, like photo or video galleries. Also i made some
      changes in the company framework, Ubiquo, to fit the project
      requeriments. \href{http://fcbarcelona.com}{http://fcbarcelona.com}
    \item xiptv. This was a video on demand project for regional
      televisions. Basically we developed a Rails application that
      offers the different videos for each program in all the
      televisions of the autonomic group.
      The interesting part was the encoding and processing (ffmpeg) of the
      master videos we received so different platforms, web and
      mobile, could watch them. \href{http://www.xiptv.cat/}{http://www.xiptv.cat/}
    \item La Vanguardia, elecciones Municipales 2011. This project was
      enterily written with Ruby (No Rails) and Javascript, without really a
      server application. This solution was really cool having all the
      site pre-generated and cached in cloudfront. \href{http://resultados-elecciones-2011.lavanguardia.com/}{http://resultados-elecciones-2011.lavanguardia.com/}
    \item lainformacion.com. Elecciones Municipales y Autonómicas
      2011. This is more or less the same project that we wrote for
      La Vanguardia but with different markup and some interesting
      features like the embeding of lainformacion.com's widgets. \href{http://elecciones.lainformacion.com/}{http://elecciones.lainformacion.com/}
    \item Comradio. I was the lead developer for this project that was
      really a big project of five application, private and public ones.
      The more interesting part of the project was the encoding and
      processing of the radio stream to generate the different tracks
      for each radio program just after they finished. \href{http://www.laxarxa.com/}{http://www.laxarxa.com/}
    \end{itemize}
    \href{http://gnuine.com}{http://gnuine.com}
    \\\href{http://github.com/gnuine/ubiquo}{http://github.com/gnuine/ubiquo}}

%------------------------------------------------

\NewEntry{2010 -- 2010}{Web developer, \textsc{Toolkom} --- Vitoria-Gasteiz}

\Description{\MarginText{Toolkom} In Toolkom we developed small web
  projects for third party companies. The technologies we used were
  Sinatra, Wordpress and Joomla. Depending on the customizations the
  client required we choose one technology or other. In many projects
  we modified some CMSs' to fit the client requeriments. The most
  important project we developed was the frontend for Gara newspaper
  web site, \href{http://www.naiz.info/}{http://www.naiz.info/}
  \\ \href{http://toolkom.com//}{http://toolkom.com/}}

%------------------------------------------------


\NewEntry{2007 -- 2009}{Research Engineer Fellow, \textsc{Univertsity
    of the Basque Country} --- Bilbao}

\Description{\MarginText{Univertsity of the Basque Country} In the
  telecomunications researching group i researched about traffic
  packages capturing in multiprocessors systems using Linux. We
  analyzed the performance of traffic processing in kernel space and in
  user space. For this we developed a processing system in kernel space
  attached to the capturing system in Linux, NAPI.
  The main objetive of this was the QoS monitoring of traffic in high
  speed networks.}
%------------------------------------------------


\NewEntry{2006 -- 2007}{System and Software Developing Fellow, \textsc{System Stein Heurtey} --- Bilbao}

\Description{\MarginText{System Stein Heurtey} As fellow in this
  company i managed a couple of Solaris systems and helped the
  employees compiling and modifing some Fortran applications the company used for
  mathematic calcs.}
%------------------------------------------------

\NewEntry{2005 -- 2006}{Web developer, \textsc{Univertsity of Basque Country} --- Bilbao}

\Description{\MarginText{Univertsity of Basque Country}During two
  years i was the developer of the website of the engineering campus
  in the Univertsity of the Basque Country. I used ASP.net and PHP for
the development.}
%------------------------------------------------

\noindent\spacedlowsmallcaps{Free Software Contributions}\vspace{1em}

\NewEntry{}{KDE desktop environment}

\Description{\MarginText{KDE}I wrote some patches in C++ for KOffice,
  the KDE office suite. I also wrote other patches for the Plasma
  desktop. I was member of KDE Spain until 2011. I gave a talk for
  other KDE developers and users in the iParty X in Castellon.}

\NewEntry{}{Wormux}

\Description{\MarginText{Wormux}Wormux is the free port of the mythic
  game, Worms. I wrote the chat for the network mode so the users
  could chat between them during the game. This development was made
  with C++ and de SDL graphical libraries.}

\vspace{1em} % Extra space between major sections
%----------------------------------------------------------------------------------------
%	EDUCATION
%----------------------------------------------------------------------------------------

\spacedlowsmallcaps{Education}\vspace{1em}

\NewEntry{2007-Present}{Univertsity of the Basque Country (EHU), Bilbao}

\Description{\MarginText{Telecomunications Engenieer. 2nd degree} Speciality
  in computer networks}

\NewEntry{2002-2006}{Univertsity of the Basque Country (EHU), Bilbao}

\Description{\MarginText{Telecomunications Engenieer, 1st degree} Speciality
  in computer networks}

%------------------------------------------------

\vspace{1em} % Extra space between major sections

%----------------------------------------------------------------------------------------
%	COMPUTER SKILLS
%----------------------------------------------------------------------------------------

\spacedlowsmallcaps{Programming Languages}\vspace{1em}

\Description{\MarginText{Actually}\textsc{Ruby}, \textsc{Javascript}}
\Description{\MarginText{Learning}\textsc{Common Lisp}, \textsc{Clojure}}
\Description{\MarginText{Others in the past}\textsc{C}, \textsc{C++}}

\spacedlowsmallcaps{Computer Skills}\vspace{1em}

\Description{\MarginText{Web Frameworks}\textsc{Ruby on Rails}, \textsc{Sinatra}, \textsc{Compojure}}

\Description{\MarginText{Testing Frameworks}\textsc{Capybara},
  \textsc{RSpec}, \textsc{TestUnit}}

\Description{\MarginText{Relational Databases}\textsc{MySQL},
  \textsc{Postgres}}
\Description{\MarginText{NoSQL Databases}\textsc{Redis}, \textsc{Riak}}

\Description{\MarginText{OS}\textsc{Arch Linux}}

\Description{\MarginText{Source Control}\textsc{git}, \textsc{bazaar}, \textsc{svn}}

\Description{\MarginText{Editors}\textsc{emacs}}

\Description{\MarginText{Amazon Services}\textsc{Cloudfront},
  \textsc{S3}, \textsc{AWS}, \textsc{RDS}}

\Description{\MarginText{Other}\textsc{RabbitMQ}, \textsc{NewRelic},
  \textsc{Airbrake/Errbit}, \textsc{Janky \& Jenkins}, \textsc{Capistrano} }


%------------------------------------------------

\vspace{1em} % Extra space between major sections

\spacedlowsmallcaps{Additional information}\vspace{1em}

\newlength{\langbox} % Create a new length for the length of languages to keep them equally spaced
\settowidth{\langbox}{English} % Length equals the length of "English" - if you have a longer language in your list put it here

\Description{\MarginText{Languages}\parbox{\langbox}{\textsc{English}}\
  \ $\cdotp$\ \ \ Fluent reading, medium in conversation}

\vspace{-0.5em} % Negative vertical space to counteract the vertical space between every \Description command

\Description{\parbox{\langbox}{\textsc{Spanish}}\ \ $\cdotp$\ \ \ Mother tongue}
%------------------------------------------------

\Description{\MarginText{Interests}Music\ \ $\cdotp$\ Guitar\ \
  $\cdotp$\ Piano \ $\cdotp$\ Ukelele \ $\cdotp$\ \ Play with new
  technologies\ $\cdotp$\ \ Food}
\end{cv}
\end{document}